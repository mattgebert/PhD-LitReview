\documentclass{article} %Report=Chapters, Article=Sections

%-----------------------PACKAGES-----------------------
\usepackage{geometry} %Allows changes of margins to document.
\usepackage{amsmath} %Allows text in equations
\usepackage{float} %Allows Placement of Images & Tables within pages, not at top etc. ie {table}[H]
%\usepackage{cite} %You want to refer to a bibtex document? But not needed with natbib.
\usepackage[super,square]{natbib}%Forces small number citations superscripted.
%\usepackage[natbibapa]{apacite} For particular citing styles?
\usepackage[dvipsnames]{xcolor} %Allows custom names and colour designs
%\usepackage{mathtools}
%\usepackage{tabularx} %Allows table widths to be set.
\usepackage[final]{pdfpages} %to insert pdf pages of code.
\usepackage{hyperref} %Hyperlinking \href{URL}{text}
\usepackage{cleveref} %Allows text like "Figure" to be automatically generated. "\cref{}" == "Tbl. \ref{}"
\usepackage{matlab-prettifier} %Matlab Code Inserts
%\usepackage{braket} %Dirac Notations
\usepackage{rotating} %Allows rotation of an image 
\usepackage{tabularx} %Allows tables that have multiple lines per cell. https://tex.stackexchange.com/questions/194737/multiline-in-a-table-cell
\usepackage{dsfont} %Allows symbols like \mathds{1} for identity matrix. 
\usepackage[justification=centering]{caption} %Allows use of subfigure environment with captions inside figure. Justification centering means multiline captions will be centred
\usepackage{subcaption} %The same as above. https://tex.stackexchange.com/questions/37581
\usepackage{listings}
\usepackage{amssymb} %Allows \therefore.
\usepackage{wrapfig} %Allows wrapping of text around figures. \begin{wrapfigure}[lineheight]{position}{width}
\usepackage{enumitem} %Allows change of horizontal seperation distances \begin{itemize}[noitemsep,topsep=0pt]
\usepackage{graphicx} %Enables use of \DeclareGraphicsExtensions command and setting graphics path.
\usepackage[toc,page]{appendix}%Allows the use of chapters within a \begin{appendices} environment.
\usepackage{multirow} %Allows the placement of multirow images in the figure environment. https://tex.stackexchange.com/a/394954/137109
\usepackage{longtable} %Allows the display of large tables over multiple pages. Used in appendix. https://www.tablesgenerator.com/latex_tables#
\usepackage[english]{babel} %Required to stop '_' causing issues in file references

%\usepackage{glossaries} %To make a glossary.  \makeglossaries  \newglossaryentry{latex}{name= , description={}}
%						\newacronym{gcd}{GCD}{Greatest Common Divisor}
\usepackage[nottoc,numbib]{tocbibind} %Adds bibliography to TOC.
\usepackage{subfiles} %Allows a TEX document to be contained within multiple TEX Files.
\usepackage{natbib}
\usepackage{outlines} %useful package to be able to do subitem enumerates and lists. use \begin{outline} with option [enumerate] instead of itemize with \1 \2 \3 as sub indexes.
\usepackage{braket} %allows use of \bra and \ket for QM.
\usepackage{esint} %allows use of \oint \oiint and other such integrals. Use \limits_{Expression} to place underneath integrals.
\usepackage{csquotes} %Allows quotations, such as \enquote, \blockcquote or \begin{dispalycquote}
\usepackage{epigraph} %Allows feature quotations which look niceish. \epigraph{Quote}{Author}



%-----------------------COMMANDS-----------------------
%Stylized quote command for features in text. 
\newcommand{\customquote}[3]{\begin{quotation} \textit{#1} \end{quotation} \begin{flushright} - #2, \textit{#3}\end{flushright} }
%Mimmic Mathematica Behaviour.
\newcommand{\pd}{\partial}
%Code to make a glossary entry:
\newcommand{\gloss}[2]{\paragraph{}\textbf{#1}	\textit{#2}}
%Code to change colour to blue of a cite.
\newcommand{\cc}{\hypersetup{citecolor=blue}}
%Code to add single equation lines to align*'s
\newcommand\numberthis{\addtocounter{equation}{1}\tag{\theequation}}
%Defines a typeset for generating 'code' text.
\definecolor{lightgrey}{RGB}{200,200,200}
\newcommand{\code}[1]{\colorbox{lightgray}{\lstinline[basicstyle=\ttfamily\color{black}]|#1|}}
%Code to add \textit{et al} text.
\newcommand{\etal}[1]{\textit{et al.}#1}
%Tensor product symbol
\newcommand{\tens}{\mathbin{\mathop{\otimes}}} 
%Trace text in equations:
\newcommand{\tr}[1]{\text{Tr}\left[#1\right]}
%Greater and Less than Similar commands simplified:
\newcommand{\gsim}{\gtrsim}
\newcommand{\lsim}{\lesssim}

%Materials:
\newcommand{\silicondioxide}[1]{SiO$_2$#1}
\newcommand{\aluminimumoxide}[1]{AlO$_3$#1}
\newcommand{\galliumoxide}[1]{GaO$_3$#1}
\newcommand{\tinoxide}[1]{SnO}
\newcommand{\bismuthoxide}[1]{Bi$_2$O$_3$#1}
\newcommand{\bismuthselinide}[1]{Bi$_2$Se$_3$#1}
\newcommand{\bismuthtelluride}[1]{Bi$_2$Te$_3$#1}
\newcommand{\antimonyselenide}[1]{Sb$_2$Se$_3$#1}
\newcommand{\antimonytelluride}[1]{Sb$_2$Te$_3$#1}
\newcommand{\cgt}[1]{Cr$_2$Ge$_2$Te$_6$#1}
\newcommand{\mercurytelluride}[1]{HgTe#1}
\newcommand{\cadmiumtelluride}[1]{CdTe#1}
\newcommand{\bismuthantimony}[1]{Bi$_{1-x}$Sb$_x$#1}



%-----------------------DOCUMENT SETUP-----------------------
%Allows display breaks in equations across pages. More condensed.
\allowdisplaybreaks
%Change the bibliography name to references:
\renewcommand\bibname{References}
%Allows custom setup of URL link colours:
\hypersetup{
	linkcolor={red!50!black},
	citecolor={blue!50!black},
	urlcolor={blue!80!black},
	colorlinks = true,
	hidelinks=true
	%	pdfborder={0 0 1}
}
\AtBeginDocument{\hypersetup{ 
		pdfborder={10 10 0.1 [1 0]}, %Required to go within "\AtBeginDocument" here otherwise overrided. {HorizontalCornerRadius VerticleCordnerRadius Thickness [Dashes PerUnits] }
		citebordercolor={0 0 1}, %{R G B} on a scale 0 to 1.
		linkbordercolor={0 0.9 0}, %{R G B} on a scale 0 to 1.	
}}
%Setup Epigraph quoting:
\setlength\epigraphwidth{.8\textwidth}
\setlength\epigraphrule{0pt}

%add specific resources
\DeclareGraphicsExtensions{.pdf,.png,.jpg,.svg}
\graphicspath{ {./graphics/}{./litreview_resources/} }

% --------------------- Document ----------------------------

\setcounter{secnumdepth}{3} %https://tex.stackexchange.com/questions/61033/setting-toc-depth-not-working
\setcounter{tocdepth}{3} %Allows increased depth into sections from defaul Chap/Sec/Subsec {2}

\font\myTitle=cmr12 at 29pt
\title{Literature Review}
\date{\today}
\author{Matthew Gebert}

\begin{document}
	\newgeometry{left=3cm, bottom=3cm}
	\maketitle
	\begingroup
	\centering
%		\scshape\Large Literature Review:\\
		\vspace{5mm}\LARGE Interfacing topological insulators surface states with ferroics\\\vspace{1.5cm}	
		%	\vspace{5mm}\Large Honours Thesis\\\vspace{1.5cm}	
	\endgroup
	\pagenumbering{arabic}
	\normalfont\normalsize
		
\section{Introduction}

\textcolor{red}{TO COMPLETE AFTER}

\section{Topological Insualtors}
\subfile{sections/topological_insulators.tex}

\section{Magnetic Materials}
\subfile{sections/magnetic_materials.tex}


\section{TI \& FI Heterostructures}
\subfile{sections/TI_&_FI_Heterostructures.tex}

\section{Experimental Methods}
\subfile{sections/experimental_methods.tex}


\section{Knowledge \& Concepts}
\subsection{Symmetry}
Symmetry is very important in all physical systems. 
For example, the order exhibited in crystals can be described through the breaking of the continuous (rotational \& translational) symmetry of space. This is due to the electrostatic interactions that cause a periodic lattice. In Magnets, spin space and time reversal symmetry are broken.


\subsection{Spintronics}
\subsubsection{Ahrimhov Bohm Effect}

\subsection{Fine Structure Constant}
The fine structure constant, also known as Sommerfields constant, is the coupleing constant "$\alpha$" that measures the strength of the EM force interacting with light.

Two methods it has been measured by include the anomalous magnetic moment of the electron, $a_e$, as well as appearing in the Quantum Hall Effect (QHE). 

It was originally introduced by Summerfieldas a theoretical correction to the Bohr  model, explaining fine structure through elliptical orbits and relitivistic mass-velocity. 

\subsection{Hamiltonians of Crystal Lattices}
Turns out that you can describe the hamiltonian of the electrons in a crystal lattice.  For this, there exists first quantisation, second quantisation hamiltonian.

\begin{itemize}
	\item First quantisation - The classical particles are assigned wave amplitudes. This is "semi-classical" where only the particles or objects are treated using quantum wavefunctions, but environment is classical. 
	\item Second quantisation (Canonical Quantisation, occupation number representation) - The wave fields are "quantized" to describe the problem in terms of "quanta" or particles. This usually means referring to a wavefunction of a state, described the the vacuum state with a series of creation operators to create the current state.
	Fields are now treated as field operators, similar to how physical quantities (momentum, position) are thought of as operators in first quantisation. 
\end{itemize}

\subsubsection{Graphene}


\section{Quotes}
\epigraph{\textit{Finally we introduce the parameter of which all the fuss is about.}}{R.F. Hofstadter, \textit{\cite{hofstadter_energy_1976}}}


\section{Questions to Answer}
	\begin{itemize}
		\item SOC - Why only in heavier elements? Explain properly. Also, why think relativistically? Motion of electron means magnetic field observed?
		\item Spin is in the 2D surface plane of a 3D TI, always perpendicular to that of the momentum - how can I couple a ferromagnetic material if there's no out-of plane?
		\item ARPES - how can this be used to verify the surface states of TI materials? (and differentiate from the bulk).
		\item Dirac Cones - why do these only appear in 2D systems.
		\item Why is spin-polarization called "Helical" spin polarization?
	\end{itemize}

\section{Glossary}
\subfile{sections/glossary.tex}

\section{Review Article Notes}
\subfile{sections/notes.tex}

\bibliographystyle{unsrtnat}
\bibliography{MyLibrary}



\end{document}