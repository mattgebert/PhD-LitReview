\documentclass{article} %Report=Chapters, Article=Sections

%-----------------------PACKAGES-----------------------
\usepackage{geometry} %Allows changes of margins to document.
\usepackage{amsmath} %Allows text in equations
\usepackage{float} %Allows Placement of Images & Tables within pages, not at top etc. ie {table}[H]
%\usepackage{cite} %You want to refer to a bibtex document? But not needed with natbib.
\usepackage[super,square]{natbib}%Forces small number citations superscripted.
%\usepackage[natbibapa]{apacite} For particular citing styles?
\usepackage[dvipsnames]{xcolor} %Allows custom names and colour designs
%\usepackage{mathtools}
%\usepackage{tabularx} %Allows table widths to be set.
\usepackage[final]{pdfpages} %to insert pdf pages of code.
\usepackage{hyperref} %Hyperlinking \href{URL}{text}
\usepackage{cleveref} %Allows text like "Figure" to be automatically generated. "\cref{}" == "Tbl. \ref{}"
\usepackage{matlab-prettifier} %Matlab Code Inserts
%\usepackage{braket} %Dirac Notations
\usepackage{rotating} %Allows rotation of an image 
\usepackage{tabularx} %Allows tables that have multiple lines per cell. https://tex.stackexchange.com/questions/194737/multiline-in-a-table-cell
\usepackage{dsfont} %Allows symbols like \mathds{1} for identity matrix. 
\usepackage[justification=centering]{caption} %Allows use of subfigure environment with captions inside figure. Justification centering means multiline captions will be centred
\usepackage{subcaption} %The same as above. https://tex.stackexchange.com/questions/37581
\usepackage{listings}
\usepackage{amssymb} %Allows \therefore.
\usepackage{wrapfig} %Allows wrapping of text around figures. \begin{wrapfigure}[lineheight]{position}{width}
\usepackage{enumitem} %Allows change of horizontal seperation distances \begin{itemize}[noitemsep,topsep=0pt]
\usepackage{graphicx} %Enables use of \DeclareGraphicsExtensions command and setting graphics path.
\usepackage[toc,page]{appendix}%Allows the use of chapters within a \begin{appendices} environment.
\usepackage{multirow} %Allows the placement of multirow images in the figure environment. https://tex.stackexchange.com/a/394954/137109
\usepackage{longtable} %Allows the display of large tables over multiple pages. Used in appendix. https://www.tablesgenerator.com/latex_tables#
\usepackage[english]{babel} %Required to stop '_' causing issues in file references

%\usepackage{glossaries} %To make a glossary.  \makeglossaries  \newglossaryentry{latex}{name= , description={}}
%						\newacronym{gcd}{GCD}{Greatest Common Divisor}
\usepackage[nottoc,numbib]{tocbibind} %Adds bibliography to TOC.
\usepackage{subfiles} %Allows a TEX document to be contained within multiple TEX Files.
\usepackage{natbib}
\usepackage{outlines} %useful package to be able to do subitem enumerates and lists. use \begin{outline} with option [enumerate] instead of itemize with \1 \2 \3 as sub indexes.

%-----------------------COMMANDS-----------------------
%Mimmic Mathematica Behaviour.
\newcommand{\pd}{\partial}
%Code to make a glossary entry:
\newcommand{\gloss}[2]{\paragraph{}\textbf{#1}	\textit{#2}}
%Code to change colour to blue of a cite.
\newcommand{\cc}{\hypersetup{citecolor=blue}}
%Code to add single equation lines to align*'s
\newcommand\numberthis{\addtocounter{equation}{1}\tag{\theequation}}
%Defines a typeset for generating 'code' text.
\definecolor{lightgrey}{RGB}{200,200,200}
\newcommand{\code}[1]{\colorbox{lightgray}{\lstinline[basicstyle=\ttfamily\color{black}]|#1|}}
%Code to add \textit{et al} text.
\newcommand{\etal}[1]{\textit{et al}#1}
\newcommand{\etals}[1]{\textit{et al's}#1}
%Tensor product symbol
\newcommand{\tens}{\mathbin{\mathop{\otimes}}} 
%Trace text in equations:
\newcommand{\tr}[1]{\text{Tr}\left[#1\right]}
%Greater and Less than Similar commands simplified:
\newcommand{\gsim}{\gtrsim}
\newcommand{\lsim}{\lesssim}

%Materials:
\newcommand{\silicondioxide}[1]{SiO$_2$#1}
\newcommand{\aluminimumoxide}[1]{AlO$_3$#1}
\newcommand{\galliumoxide}[1]{GaO$_3$#1}
\newcommand{\tinoxide}[1]{SnO}
\newcommand{\bismuthoxide}[1]{Bi$_2$O$_3$#1}
\newcommand{\bismuthselinide}[1]{Bi$_2$Se$_3$#1}


%-----------------------DOCUMENT SETUP-----------------------
%Allows display breaks in equations across pages. More condensed.
\allowdisplaybreaks
%Change the bibliography name to references:
\renewcommand\bibname{References}
%Allows custom setup of URL link colours:
\hypersetup{
	linkcolor={red!50!black},
	citecolor={blue!50!black},
	urlcolor={blue!80!black},
	colorlinks = true,
	hidelinks=true
	%	pdfborder={0 0 1}
}
\AtBeginDocument{\hypersetup{ 
		pdfborder={10 10 0.1 [1 0]}, %Required to go within "\AtBeginDocument" here otherwise overrided. {HorizontalCornerRadius VerticleCordnerRadius Thickness [Dashes PerUnits] }
		citebordercolor={0 0 1}, %{R G B} on a scale 0 to 1.
		linkbordercolor={0 0.9 0}, %{R G B} on a scale 0 to 1.	
}}
%add specific resources
\DeclareGraphicsExtensions{.pdf,.png,.jpg,.svg}
\graphicspath{ {./graphics/}{./litreview_resources/} }

% --------------------- Document ----------------------------

\setcounter{secnumdepth}{3} %https://tex.stackexchange.com/questions/61033/setting-toc-depth-not-working
\setcounter{tocdepth}{3} %Allows increased depth into sections from defaul Chap/Sec/Subsec {2}

\font\myTitle=cmr12 at 29pt
\title{Literature Review}
\date{\today}
\author{Matthew Gebert}

\begin{document}
	\newgeometry{left=3cm, bottom=3cm}
	\maketitle
	\begingroup
	\centering
%		\scshape\Large Literature Review:\\
		\vspace{5mm}\LARGE Topological insulators surface states with ferroics\\\vspace{1.5cm}	
		%	\vspace{5mm}\Large Honours Thesis\\\vspace{1.5cm}	
	\endgroup
	\pagenumbering{arabic}
	\normalfont\normalsize
		
		
\section{Topological Insualtors}
\subsection{Introduction}
\subsection{Spin-Orbit Coupling}
Doesn't break time-reversal symmetry like magentic field for QHE, but can lead to \textbf{quantum spin hall effect}, or QSHE, where electrons differentiated by their spin move in opposite directions. No conserved spin current to test in a material.

\subsection{Material Growths}

\section{Magnetic Materials}
\subsection{Classifications}
\subsubsection{Ferromagnetic}
\subsubsection{Anti-ferromagnetic}
\subsubsection{Paramagnetic}
\subsubsection{Diamagnetic}
\subsection{Ferromagnetic Insualtors}

\section{TI \& FI Heterostructures}
\subsection{Theory}
\subsection{Experiments}
https://www-nature-com.ezproxy.lib.monash.edu.au/articles/nature13534
https://advances.sciencemag.org/content/5/8/eaaw3415
https://tms16.sciencesconf.org/data/pages/TI\_lecture2.pdf
https://arxiv.org/pdf/1401.0848.pdf


\section{Experimental Methods}

\subsection{Synthesis, Growth and Fabrication}
\subsubsection{MBE Growths}

\subsection{Metrological Methods}
\subsubsection{ARPES}
For material characterisation, angle resolved photo-emission spectroscopy (ARPES) is a method of resolving the ejection of electrons from a material by high energy photons. Analysis of the momentum of the incident and emission resolve the electronic band structure. 




\section{}


\section{Knowledge \& Concepts}
\subsection{Symmetry}
Apparently ssymmetry is very important in all physical systems. 
For example, the order exhibited in crystals can be described through the breaking of the continuous (rotational \& translational) symmetry of space. This is due to the electrostatic interactions that cause a periodic lattice. In Magnets, spin space and time reversal symmetry are broken.
%


\subsection{Spintronics}
\subsubsection{Ahrimhov Bohm Effect}

\subsection{Fine Structure Constant}
The fine structure constant, also known as Sommerfields constant, is the coupleing constant "$\alpha$" that measures the strength of the EM force interacting with light.

Two methods it has been measured by include the anomalous magnetic moment of the electron, $a_e$, as well as appearing in the Quantum Hall Effect (QHE). 

It was originally introduced by Summerfieldas a theoretical correction to the Bohr  model, explaining fine structure through elliptical orbits and relitivistic mass-velocity. 

\subsection{Hamiltonians of Crystal Lattices}
Turns out that you can describe the hamiltonian of the electrons in a crystal lattice.  For this, there exists first quantisation, second quantisation hamiltonian.

\begin{itemize}
	\item First quantisation - The classical particles are assigned wave amplitudes. This is "semi-classical" where only the particles or objects are treated using quantum wavefunctions, but environment is classical. 
	\item Second quantisation (Canonical Quantisation, occupation number representation) - The wave fields are "quantized" to describe the problem in terms of "quanta" or particles. This usually means referring to a wavefunction of a state, described the the vacuum state with a series of creation operators to create the current state.
	Fields are now treated as field operators, similar to how physical quantities (momentum, position) are thought of as operators in first quantisation. 
\end{itemize}

\subsubsection{Graphene}


\section{Review Article Notes}
\subsection{J. Moore - The Birth\cite{moore_birth_2010}}
\subsubsection{Lessons from the past}
\begin{itemize}
	\item Quantum Hall Effect can occur in 2D systems (condined) in the presence of a magentic field. It is a consequence of topological properties of the electronic wavefunctions. 
	\item Question - Can this effect arise without a large external magnetic field?
	\item In the 1980s, Predicted that forces from motion through crystal lattice could provide the same hall state \cite{haldane_model_1988}
	\item The mechanism which this recently has occured through is \textbf{spin-orbit coupling} or SOC. Spin and orbital angular momentum degrees of freedom are coupled. This coupling causes electrons to feel a spin dependent force.
	\item QSHE predicted in 2003. Unclear how to measure or if realistic or not. \cite{murakami_dissipationless_2003, murakami_spin-hall_2004}
	\item Kane \& Mele in 2005 produced a key theory advance, using realistic models. Showed that QSHE can survive by the use of invariants and could compute if the 2D material has an edge state or not. 2D Insulators that have 1D wires that conduct perfectly at low temps, similar to QHE. \cite{kane_z_2005}
	\item (Hg,Cd)Te quantum wells predicted to have quantized charge conductance  \cite{bernevig_quantum_2006}. These were then picked up in 2007 \cite{konig_quantum_2007}.
\end{itemize}
\subsubsection{Going 3D}
\begin{outline}
	\1 2006 gave the revelation that while QHE doesn't go 2D to 3D, the TI does, subtly. \cite{fu_topological_2007, moore_topological_2007, roy_topological_2009}
	\1 "Weak 3D TI" given through stacking of multiple 2D TIs. Not stable to disorder. A dislocation will always conains a quantum wire at the edge.
	\1 "Strong 3D TI", connects ordinary insulators and topological insualtors by breaking time-reversal symmetry. \cite{moore_topological_2007} This is the focus result for experimental physics at the moment.
	\1 SOC is also required, but mixes all spins, unlike the 2D case. Spin direction dictates the momentum along the surface. Scattering also occurs, but metallic surface cannot vanish.
	\1 First 3D TI was Bi$_x$Sb$_{1-x}$. Surfaces mapped using ARPES experiment. The surface was complex, but launched a search for other TIs. \cite{hsieh_topological_2008,hsieh_observation_2009}.
	\1 Heavy metal, small bandgap semiconductors are meant to be the best candidates (this sounds like TMDs haha but not implied) for two reasons.
		\2 SOC is relativistic, and only strong for heavy elements.
		\2 Large bandgaps, relative to the scale of SOC, will not allow SOC to change the phase. This is because bands either need to invert or be able to cross or change properties by local deformation (I think).
	
	\1 The discovery of Bismuth Selenide Bi$_2$Se$_3$ and Bismuth Telluride Bi$_2$Te$_3$. They exhibit TI behaviour to higher temperatures, bulk bandgaps of $>$ 0.1 eV \cite{xia_observation_2009,zhang_topological_2009,chen_experimental_2009}.
 		\2 The large bandgap is good for measurement in higher temperature
 		\2 Simple surface states to investigate heterostructures etc. 
 		\2 Complication - Distinction between surface and bulk states - residual conductivity resulting from impurities.
	
	\1 Graphene? \bismuthselinide{} looks like Graphene, due to it's Dirac nature. The difference for graphene is that a TI has only one Dirac point. Graphene has spin degeneracy with two Dirac points. The difference is major, including applications for quantum computing. This implies that the bandstructure \& reciporical lattice are really important for preserving coherent spin phenomena.
	
	\2 STMs found interference patterns near steps / defects, showing electrons aren't completely reflected - even when strong disorder occurs \cite{roushan_topological_2009,alpichshev_stm_2010, zhang_experimental_2009}. 
	\3 Generally, \textit{anderson localization}, the formation of insulating states, is meant to occur in materials with surface states (ie, Noble metals).
	\3 TI surface states are distinguished through this property! 
	\3 In some 2D models, it seems to form as a result of some disorder?
	\3 In graphene, apparently smooth potentials might do the same, but it's quite unlikely!
	\2 Femi level is located \textbf{at} the Dirac point for graphene! This is very adventageous, because of the tunability of the material. However, when it comes to other materials, it will not be as easy to tune the DOS around some point of interest like in graphene, without some form of doping. It was demonstrated recently in \bismuthselinide{} that you could tune to the Dirac point as well \cite{hsieh_tunable_2009}. The chemical potential is really important to be able to control for purposes like this. 	
	
	\end{outline}

	\subsubsection{Materials Challenges}
	\begin{outline}
		\1 Two important experiments:
			\2 Aharonov-Bohm Effect observed in nano-ribbons
			\2 MBE grow thin films of \bismuthselinide{} controlling thickness.
		\1 Items to characterize:
			\2 Transport \& optical properties on the metalic surface to measure the spin state and conductivity.
			\2 Improved materials with reduced bulk conductivity may be required!
	\end{outline}

	\subsubsection{A Nascent Field}
	\begin{outline}
		\1 Particle physics in the 1980s predicted a particle who coupled to ordinary EM fields.
		\1 Cond.Mat Physics seek phases of matter to respond to influence. 
			\2 Solid in response to sheer force
			\2 Superconductor in response to Meissner effect
			\2 T.I. in response to axion electrodynamics, ie in response to a scalar product $\textbf{E}.\textbf{B}$. (1980s)
				\3 Two coefficients correspond to OIs (ordinary) and TIs.
		\1 Just like insulators modify dielectric constant, coefficient of $\textbf{E}^2$ in Lagrangian, \\TI's generate non-zero coefficient of $\textbf{E.B}$
		\1 Applications are in electronics and magnetic memeory, not due to speed, but rather due to reproducability and no fatigue, as a result purely from the electrons.
		\1 Apparently, you can determine if a material is topological or not, purely from a polarizing magnetic field measurement.
		\1 The magnetoelectric effect can also create an electric field from a magnetic source, and vice versa. 
		
		\1 Some work also done to find similar phases, in materials with differing symmetries. Cooper pairs in superfluids ($^3$He) are closely connected to the quasiparticles/electrons in topological insulators. Direct experimental signatures would be awesome.
		
		\1 Moore goes on to talk about the interface of a superconductor and a topological insulator:
			\2 Creation of emergent particles
			\2 TI surface becomes superconducting | Proximity effect
				\3 Vortex line from SC into the TI, a 0-energy Majorana fermion is trapped.
				\3 Majorana fermions have different quantum numbers to that of a regular electron.
				\3 Is roughly 'half' of an electron,  and it's own antiparticle.
			\2 Direct observation of Majorana fermions is a long sought goal
			\2 New particle statistics for other quasiparticles and systems. 
	\end{outline} 

\subsection{The Quantum Spin Hall Effect \& Topological Insulators \cite{qi_quantum_2010} - Physics Today (2010)}
	\subsubsection{Intro}
	\begin{outline}
	\1 Phases \& phenomena defined by symmetry	
		\2 Translation = Crystal Solids
		\2 Rotational = Magents
		\2 Gauge = Superconductors
	\1 QHE first example of non-spontaneously broken symmetry.
		\2 Topologically defined, not geometrically!
	\1 SQHE states are distinct again. 
		\2 Immune to impurities or geometric pertubations (momentum spin locking, scattering is supressed)
		\2 Maxwells equations altered by an additional 'topolgical' term, with remarkable effects. 
	\1 QHE differs to QSH
		\2 QHE External magnetic field required, TR symmetry broken
		\2 QSHE TR symmetric.
	\end{outline}
	\subsubsection{QH to QSH}
	\begin{outline}
	\1 Traffic control seperating out lanes of movement provides much less resistance. Same for QSH.
	\1 QHE Magnetic field pushes electrons to side lanes.
		\2 there are only two degrees of freedom in the system (forward above and back below)
		\2 There is no way for electrons to turn around
		\2 limited by external Magnetic Field requirement
	\1 In a real 1D system, you have four channels - spin up and down, moving forward and backward
		\2 These were predicted to be able to split into 2 separate chanels on two sides of the material. 
	\1 The backscattering supression is quantum mechanical.
		\2 Electrons in edge states can be thought to circle around the edge of the impurity and scatter backwards.
		\2 Depending on which way they rotate, they pick up a phase of $\pi$ or $-\pi$ (imagine a vector rotating around a circle). 
		\2 This induces a $2\pi$ phase difference between the two bodies. 
		\2 In QM, this corresponds to a negative sign. This means their amplitudes destructively interfere, leading to perfect transmission instead.
		\2 If the impurity is magnetic, TR symmetry is broken, and there is no longer destructive interference.
		\2 In an unseparated system (two movers back and foward) then spin doesn't need to change phase and regular scattering can take place.
		\2 The consequence then is, that there needs to be an "odd" number of forward and backward movers.
		\2 This is the heart of the $Z_2$ topological quantum number, and why a QSH insulator is also referred to as a topological insulator.
	\1 Requirements for 2D TIs
		\2 Heavy elements produce coupling of spin and orbital motion, relativistic.
	\1 Bernevig, Hughes and Zhang proposed a mechanism for finding TIs, nanoscopic layers sandwiched between other materials, become TIs after a critical thickness $d_c$. 
		\2 Mechanism is "band inversion", where general ordering of valence and coduction band is inverted by SOC.
		\2 Generally, \textit{s} orbitals contribute to conduction band, and valence is from the \textit{p} band.
		\2 In Hg and Te, SOC pushes S band above, and P band below.
		\2 Sandwiched between with Cadmium Telluride (similar lattice constant, but VERY different SOC).
		\2 Changing the thickness parameter changes the overall SOC of the quantum well. 
		
	
	\end{outline}


\section{Questions to Answer}
	\begin{itemize}
		\item SOC - Why only in heavier elements? Explain properly. Also, why think relativistically? Motion of electron means magnetic field observed?
	\end{itemize}

\bibliographystyle{unsrtnat}
\bibliography{MyLibrary}



\end{document}