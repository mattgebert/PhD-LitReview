\documentclass[../mattg_ti-fi_lit-review.tex]{subfiles}

There has been much interest in creating spintronic devices for modern applications. One focus in particular is a heterostructure (two or more layers) that creates a spin-torque device. That is, some ferromagnetic layer that is switchable by running an electric current through another layer. Researchers have been trying to make these devices a reality using a heterostructure of a topological insulator and ferromagnetic materials for about 10 years, since the first suggestions \cite{moore_birth_2010} of such applications of TIs.

\subsection{Quantum Anomalous Hall Effect}
The anomalous hall effect been observed in ferromagnetic materials for a long time. It's anomalous because it appears without the need of an external magnetic field. The origins for the effect can be both intrinsic, or extrinsic (system dependent disorder). 

It turns out that just like the Hall effect and the spin Hall effect have quantised versions, there also exists a quantum anomalous Hall effect (QAHE). 

\subsection{Spin Torque}
\subsubsection{Torque}
\subsubsection{Spin Torque Angle - Spin Torque Efficiency}
\subsubsection{Magnetization}
Larmor Equation:
\begin{align}
	\vec{T} = \mu_0 \vec{\textbf{M}}\times \vec{\textbf{H}}
\end{align}
Torque:
\begin{align}
	\frac{d\vec{L}}{dt}=\vec{T}
\end{align}
Gyromagnetic Ratio
\begin{align}
	\gamma\equiv\frac{\mu}{L}
\end{align}
Magnetization Change
\begin{align}
\frac{d\vec{M}}{dt} = \gamma\vec{T}
\end{align}
\subsubsection{Critical Current Density}
\subsubsection{Charge Spin Conversion Efficiency}
\subsection{Experiments}
\textcolor{red}{TODO: go through past experiments. Take significant of results, sample fabrication, experimental methodology}.
\subsubsection{Conductive films (non-insulating)}
\begin{itemize}[noitemsep,topsep=0pt,parsep=0pt,partopsep=0pt]
	\item (2012) Liu et al. Beta-Ta with CoFeB  (Observed Torque, 2.1mT for 0.7mA, Sputtered/evaporated onto surfaces) 
	\item (Apr 2013) Chang et al. Observation of QAHE - Cr0.15(Bi0.1Sb0.9)1.85Te3  - 5 quintuple layers, on SrTiO3, grown through MBE. Mobilities lower than 1000cm2/Vs. Chromium doping to make ferromagnetic, RHEED, ARPES.
	\item (Apr 2014) Y. Fan et al. Epitaxial (Bi0.5Sb0.5)2Te3/(Cr0.08Bi0.54Sb0.38)2Te3 bilayer films, 3 \& 6 Quintuple layers respectively. AHE. ($\theta$SH reported to be very large??). Look again at device fab. 
	\item (2014) A.R. Mellnik et al.  8nm Bi2Se3 and 8nm / 16nm Ni81Fe19 Bilayer (MBE \& Sputtering) (Observed Torque Applied, Torque per unit moment Tparallel = 2.7 x 10-5, T¬perp=3.7 x 10-5), 
	$\omega$SH\_parallel=(2.0 ± 0.4) × 105/2e $\omega$-1 m-1 and  $\sigma$SH\_perp  = (1.6 ± 0.2) × 105/2e $\omega$-1 m-1
	\item  (2016) P. Li et al. Pt/BaFe12O19 – Torque observed, value not mentioned. 
 	\item (Aug 2017) J. Han et al. (MIT \& Samarth Grp) - GaAs / Bi2Se3(7.4nm)/CoTb(4.6nm)/SiNx(3 nm) - Magnetron and MBE. Critical current density 2.8 x 106 A/cm$^2$. 
	0.16 effective SH angle. Large discrepancies noted in this paper. Also searched (Bi, Sb)2Te3, and compared to Pt and other heavy metals.
	\item (Sep 2017) K. Yasuda et al. Crx(Bi1-ySby)2-xTe3/(Bi1-ySby)2Te3 - Magnetic non-magnetic heterostructures. Second Harmonic Voltage goverened by asymmetric magnon scattering gives incorrect value of CSC (Charge spin conversion efficiency).
	\item \textbf{(Nov 2017) Y. Wang et al. Bi2Se3/NiFe Heterostructure, Describes contamination of BS, 2DEG to the SOT effects from TSS affecting Bi2Se3 systems. Also used Bi2Se3/Co40Fe40B20, and Bi2Se3 (8 QL)/Py(6nm). Growths on Al2O3. Used MgO and Al2O3 to protect. Also SiO2 used for Py devices. Iron Milling used with Photolithography.}
	\item (Jul 2018) Khang et al. Bi0.9Sb0.1, 5/10nm Mn0.45Ga0.55 3nm layers\\ $\sigma\approx2.5\times10^5\omega^{-1}m^{-1},\theta_{SH}\approx 52$ and $\theta_{SH}\approx1.3\times10^7\hbar / 2e \omega^{-1}m^{-1}$
	\item (Jul 2018) Mahendra DC et al. BixSe(1-x)/Co20Fe60B20 (Magentron Sputtered)\\
	$\theta_S = 18.62$ (d.c. planar Hall) \&8.67 (ST-FMR) [Si/SiO2/MgO (2nm)/BixSe(1-x)(tBS nm)/ CoFeB(5nm)/MgO(2nm)/Ta(5nm)] tBS=4,6,8,16,40nm
	\item (Jan 2018) Y. Li et al.  Epitaxial (001) SmB6/Si thin films (50nm +) with MgO(1.5)/CFB(Co40Fe40B20, 1)/$\beta-W(0.8)$ – DC Magnetron sputtering / DC\&RF magnetron…
\end{itemize}
\subsubsection{Insulating Films}
\begin{itemize}[noitemsep,topsep=0pt,parsep=0pt,partopsep=0pt]
	\item (2019) P. Li et al. Bi2Se3 and FI - BaFe12O19 \url{https://advances.sciencemag.org/content/5/8/eaaw3415}
	\item YIG (2016) H. Wang et al. \url{https://journals-aps-org.ezproxy.lib.monash.edu.au/prl/pdf/10.1103/PhysRevLett.117.076601}
\end{itemize}

\subsection{Non-TI Devices}
- (2010) Co layer 0.6nm, with Pt 3nm and AlO\_x layers 2nm, transverse magnetic fields. https://www-nature-com.ezproxy.lib.monash.edu.au/articles/nmat2613?proof=trueMay

