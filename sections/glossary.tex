\documentclass[../mattg_ti-fii_lit-review.tex]{subfiles}

\begin{document}
	\subsection{B}
	\begin{itemize}
		\item 
			\textbf{Berry Connection} is the bra-ket of a Gradient operator of vector $\vec{R}$ acting on a wavefunction over some path  $\Gamma$ with positions $\vec{R}$. Physically it is the mechanism from getting one geometric space point to another, ie the change of parameters.
			\begin{equation}
				\vec{\mathcal{A}_n}(\vec{R}) = i\braket{\psi_n(\vec{R})|\nabla_{\vec{R}}|\psi_n(\vec{R})}
			\end{equation}
			This object has N parts, for each of the vector components. There's a different connection for every eigenstate, of every point in the space. 
			It's a little more subtle than a vector - it transforms under Gauge transformations. For an example wavefunction to include some additional phase factor (i.e. Gauge transformation) $\ket{\psi_n'(\vec{R})}=e^{-i\beta(\vec{R})}\ket{\psi_n(\vec{R})}$
			\begin{equation}
				\vec{\mathcal{A}_n'}(\vec{R}) = \vec{\mathcal{A}_n}(\vec{R}) + \nabla_{\vec{R}}\beta(\vec{R})
			\end{equation}
			So the "connection" is that it transforms with the gradient of a function, like a vector potential.
		\item 
			\textbf{Berry Curvature} is the consequence or result of going from one set of starting parameters to the same set of parameters over some path (also called a \textbf{connection}). You could use the Schroedinger equation to provide a connection. It turns out that through evolution the state can pick-up a phase factor, relative to the original starting state. This phase factor is also known as the Berry phase, and has consequences for the quantum mechanical properties of the system.
		\item 
			\textbf{Berry Phase} is the integral
			\begin{align}
				\gamma_n(\Gamma) &= \int_\Gamma \vec{\mathcal{A}_n}(\vec{R})\cdot d\vec{R}\\
				\text{Changing gauge: }\gamma'_n(\Gamma) &= \int_\Gamma \vec{\mathcal{A}_n}(\vec{R})\cdot d\vec{R} + \int_\Gamma \nabla_{\vec{R}}\vec{\beta} \cdot d\vec{R}\\
				\implies \gamma'_n(\Gamma) &= \gamma_n(\Gamma) + \beta(R_f) - \beta(R_i)
			\end{align}
			It's not Gauge invariant, which makes it difficult to be useful for different measurements. However, if you begin \textbf{AND} end in the same configuration point (ie, a closed loop) then this value is Gauge invariant!
			\begin{itemize}
				\item Eigenstates have to be imaginary to achieve a non-zero Berry phase.
				\item If the path is 1D, then the integration cancels out to result in a zero Berry phase, again.
				\item In 3D, the Berry phase over some path $\Gamma$ enclosing a surface $S$, then Stokes theorem lets you specify the \textbf{Berry curvature} 
				\begin{align}
					\oint_\Gamma \vec{\mathcal{A}_n}\cdot d\vec{R} &= \oiint_S \left(\nabla\times\vec{\mathcal{A}}\right)\cdot d\vec{a}\\
					\vec{D}&=\nabla_{\vec{R}}\times\vec{\mathcal{A}}(\vec{R})
				\end{align}
			\end{itemize}
	\end{itemize}
		
\end{document}