\documentclass[../mattg_ti-fi_lit-review.tex]{subfiles}

This review of existing literature is to support my Ph.D. thesis: experimentally probing the electronic transport of interfaces between topological insulators and ferroics. I will use this document to review the field of topological insulators, ferromagnetic insulators and resulting heterostructures. 

In the last two decades there has been a vast discovery of new materials that exhibit phenomenal properties interesting for condensed matter physics research. Usually these materials are stumbled upon experimentally and explained by theorists. In the case of discovering the first topological materials since the quantum Hall effect, new insights in theory drove significant experimental results.

Topological insulators (TIs) attract interest because of their unique properties. 2D and 3D TIs have edge and surface electronic states respectively, and these states exhibit useful properties such as momentum-spin locking (or spin-polarised current densities) and suppressed back-scattering, as well as exhibit Dirac fermions, similar to graphene. These properties may be used for applications which require directing currents that are spin-polarised as well as robust to disorder, with low resistance.

This work ties into the research themes of FLEET (Future Low Energy Electronic Technologies) and the government's and FLEET's funding goals to enable new, low energy computing technology.

%TODO: Talk about general origins of the field of TIs. Lead into hall effect.
%\textcolor{red}{TO COMPLETE AFTER WRITING REVIEW}