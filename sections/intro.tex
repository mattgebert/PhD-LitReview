\documentclass[../mattg_ti-fi_lit-review.tex]{subfiles}

In the last two decades there has been a vast discovery of new materials that exhibit phenomenal properties interesting for condensed matter physics research. Usually these materials are stumbled upon experimentally, however in the case of topological materials the setting and expectation was purely theoretical.

Topological insulators (TIs) attract interest because of their unique properties related to electronics. 2D and 3D compounds have edge and surface electronic states respectively, and these states exhibit useful properties such as momentum-spin locking (or localised spin densities) and suppressed back-scattering.

These properties may be used for applications like directing currents that are spinful, ie, exerting magnetic fields or used in spintronics, as well as providing currents that are robust with low resistance due to ensured conductivity.

This literature review aims to walk through the basics of TI materials

%TODO: Talk about general origins of the field of TIs. Lead into hall effect.
\textcolor{red}{TO COMPLETE AFTER WRITING REVIEW}