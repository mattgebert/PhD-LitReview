\documentclass[../mattg_ti-fi_lit-review.tex]{subfiles}

\subsection{Symmetry}
Symmetry is very important in all physical systems. 
For example, the order exhibited in crystals can be described through the breaking of the continuous (rotational \& translational) symmetry of space. This is due to the electrostatic interactions that cause a periodic lattice. In Magnets, spin space and time reversal symmetry are broken.

\subsubsection{Symmetry points}
In solid state physics, when observing the reciprocal space for the dispersion relations, there exist particular symmetry points, such as the $\Gamma$ point, $K$ points, $M$ points and so forth. These have importance in the behaviour of materials due to their propagation vectors through the crystal.

\textcolor{red}{Not sure if this is entirely correct.}
\begingroup
\allowdisplaybreaks
\begin{table}[H]
	\centering
	\begin{tabular}{ll}
		\hline
		\multicolumn{2}{|c|}{\textbf{General}}                                                                      \\ \hline
		\multicolumn{1}{|l|}{$\Gamma$} & \multicolumn{1}{l|}{Center of the Brillouin zone}                                 \\ \hline
		\multicolumn{2}{|c|}{\textbf{Simple cube}}                                                                  \\ \hline
		\multicolumn{1}{|l|}{M} & \multicolumn{1}{l|}{Center of an edge}                                            \\ \hline
		\multicolumn{1}{|l|}{R} & \multicolumn{1}{l|}{Corner point}                                                 \\ \hline
		\multicolumn{1}{|l|}{X} & \multicolumn{1}{l|}{Center of a face}                                             \\ \hline
		\multicolumn{2}{|c|}{\textbf{Face-centered cubic}}                                                          \\ \hline
		\multicolumn{1}{|l|}{K} & \multicolumn{1}{l|}{Middle of an edge joining two hexagonal faces}                \\ \hline
		\multicolumn{1}{|l|}{L} & \multicolumn{1}{l|}{Center of a hexagonal face}                                   \\ \hline
		\multicolumn{1}{|l|}{U} & \multicolumn{1}{l|}{Middle of an edge joining a hexagonal and a square face}      \\ \hline
		\multicolumn{1}{|l|}{W} & \multicolumn{1}{l|}{Corner point}                                                 \\ \hline
		\multicolumn{1}{|l|}{X} & \multicolumn{1}{l|}{Center of a square face}                                      \\ \hline
		\multicolumn{2}{|c|}{\textbf{Body-centered cubic}}                                                          \\ \hline
		\multicolumn{1}{|l|}{H} & \multicolumn{1}{l|}{Corner point joining four edges}                              \\ \hline
		\multicolumn{1}{|l|}{N} & \multicolumn{1}{l|}{Center of a face}                                             \\ \hline
		\multicolumn{1}{|l|}{P} & \multicolumn{1}{l|}{Corner point joining three edges}                             \\ \hline
		\multicolumn{2}{|c|}{\textbf{Hexagonal}}                                                                    \\ \hline
		\multicolumn{1}{|l|}{A} & \multicolumn{1}{l|}{Center of a hexagonal face}                                   \\ \hline
		\multicolumn{1}{|l|}{H} & \multicolumn{1}{l|}{Corner point}                                                 \\ \hline
		\multicolumn{1}{|l|}{K} & \multicolumn{1}{l|}{Middle of an edge joining two rectangular faces}              \\ \hline
		\multicolumn{1}{|l|}{L} & \multicolumn{1}{l|}{Middle of an edge joining a hexagonal and a rectangular face} \\ \hline
		\multicolumn{1}{|l|}{M} & \multicolumn{1}{l|}{Center of a rectangular face}                                 \\ \hline                                                                
	\end{tabular}
\end{table}
\endgroup


\subsection{Fine Structure Constant}
The fine structure constant, also known as Sommerfields constant, is the coupleing constant "$\alpha$" that measures the strength of the EM force interacting with light.

Two methods it has been measured by include the anomalous magnetic moment of the electron, $a_e$, as well as appearing in the Quantum Hall Effect (QHE). 

It was originally introduced by Summerfieldas a theoretical correction to the Bohr  model, explaining fine structure through elliptical orbits and relitivistic mass-velocity. 

\subsection{Hamiltonians of Crystal Lattices}
Turns out that you can describe the hamiltonian of the electrons in a crystal lattice.  For this, there exists first quantisation, second quantisation hamiltonian.

\begin{itemize}
	\item First quantisation - The classical particles are assigned wave amplitudes. This is "semi-classical" where only the particles or objects are treated using quantum wavefunctions, but environment is classical. 
	\item Second quantisation (Canonical Quantisation, occupation number representation) - The wave fields are "quantized" to describe the problem in terms of "quanta" or particles. This usually means referring to a wavefunction of a state, described the the vacuum state with a series of creation operators to create the current state.
	Fields are now treated as field operators, similar to how physical quantities (momentum, position) are thought of as operators in first quantisation. 
\end{itemize}