

\subsection{Introduction}
In the last two decades there has been a vast discovery of new materials that exhibit phenomenal properties interesting for condensed matter physics research. Usually these materials are stumbled upon experimentally, however in the case of topological materials the setting and expectation was purely theoretical.

Topological insulators (TIs) attract interest because of their unique properties related to electronics. 2D and 3D compounds have edge and surface electronic states respectively, and these states exhibit useful properties such as momentum-spin locking (or localised spin densities) and suppressed back-scattering (lower resistance, ensured conductivity).

%TODO: Talk about general origins of the field of TIs. Lead into hall effect.

\subsection{Quantum Hall Effect}\label{sec:QHE}

The quantum hall effect (QHE) was first discovered in a 2D semiconductors (created via the interface of two semiconductors) %TODO:Cite Klitzing et all 1980.
At low temperatures, localisation of electrons in combination with Landau quantisation (the quantisation of cyclotron orbits for charged particles in magnetic fields \footnote{The reason for cyclotron orbits is due to the single valued electron wavefunction, ie $\oint \vec{P}\cdot d\vec{r} = 2\pi N$}) , leads to vanishing conductivity $\rho_{xx}$ and quantised Hall conductivity $\rho_{xy}$.  %TODO
What makes this fascinating is that the QHE is a clear example of macroscopic quantum phenomena. Other such phenomena include Bose Einstein Condensation, where many atoms condense into the same uniform quantum state, and superconductivity where the pairing of electrons (Cooper pairs) also produce a new phase of macroscopic quantum behaviour. %TODO check this...

It turns out that the QHE is also topological. Thouless (who later received the noble prize in 2016, before passing away in 2019), Kohmoto, Nightingale, and den Nijs (TKNN) recognised that \textbf{K} space maps to a non-trivial hilbert space for the QHE; the space has a topology. This topology can be specified by an integer $\nu$ which also corresponds to the hall conductance, 

\begin{equation} \label{eqn:qhe-conductivity}
	\sigma_{xy}=\nu \frac{e^2}{h}
\end{equation}


\textcolor{red}{Show calculation of berry phase of the bloch wave function calculated around the BZ boundary divided by 2 pi} %TODO

For the QHE to occur, the Landau quantisation opens gaps in the band structure of the material, and the chemical potential is situated within this gap. In a classical picture, the boundaries of the material cannot sustain these cyclotron orbit, because of the phase transition to an ordinary insulator, rather than the QHE insulator, and so edge channels open in a \textbf{bulk-boundary correspondence}. 

Note that the QHE does not preserve \textbf{time reversal} (TR) \textbf{symmetry}, as charge carriers experience different forces due to the magnetic field when their direction is inverted. The magnetic field breaks TR symmetry. You can see this in the two degrees of freedom in the system; For an out of plane magnetic field, charge is separated into two lanes, and those channels move a particular direction, ie forward above for electrons, backward below for holes. Reversing the direction of carriers yields the same channels but switched, forward above for holes, backward below for electrons. 

\subsection{Quantum Spin Hall Effect}\label{sec:QSHE}
The spin hall was observed in 2004 by Kato \etal{} \cite{kato_observation_2004}, where for particular semiconductors a spin density was observed rather than a charge density, like in the regular hall effect.
The origins for such an phenomenon are similar to that of the \hyperref[sec:QAHE]{quantum anomalous Hall effect} in ferromagnets, sometimes being extrinsic, sometimes intrinsic. Intrinsically, the \textbf{Berry curvature} of the electronic valence-band Bloch wave functions result in spin-Hall insulators. The spin Hall insulator was proposed by Zhang, where the \textbf{Berry phase} is finite implying a finite spin Hall conductivity \cite{murakami_spin-hall_2004}. While this idea did not realise spin currents, it allowed further exploration to find it quantised version, a quantum spin Hall (QSH) insulator \cite{bernevig_quantum_2006, kane_z_2005,kane_quantum_2005}.

In a real 1D system with spin, you have a regular four degrees of freedom; Spin can move either direction, and can move forward and backward. This could be separated into two copies of the QH system - spin up moving forward with spin down moving backward at the top, and spin down moving forward with spin up moving backward on the bottom. This is referred to the quantum spin Hall effect (QSHE).
But what is the key ingredient that allows this separation of states to occur? In the QHE, it's the magnetic field that breaks time reversal symmetry. For QSH insulators, the essential ingredient is \hyperref[sec:SOC]{spin orbit coupling} (SOC).

\subsection{Spin-Orbit Coupling}\label{sec:SOC}
Spin-orbit coupling (SOC) is coupling of both spin and orbital angular momentum properties of particles. Spin provides a magnetic moment, and the motion of the charged electron through its orbital angular momentum also provides a magnetic moment. From the electrons point of view, it can be considered to be the interaction of the electron's spin with the orbital motion of the nucleus.

This coupling causes splitting, similar to Zeeman splitting, of electronic energy levels. For different atoms, the SOC is different, with the trend that heavier atoms have a larger SOC \cite{herman_relativistic_1963}. 

This type of coupling doesn't break time-reversal symmetry like magentic field for QHE, but can lead to the \hyperref[sec:QSHE]{QSHE}, where electrons differentiated by their spin move in opposite directions.

\subsubsection{Rashba Effect}

\subsubsection{Dresselhaus Effect}





\subsection{Quantum Anomalous Hall Effect}\label{sec:QAHE}



\subsection{Band inversion}

\subsection{2D Dirac Systems}

\subsection{Materials \& Growths}
\subsubsection{\bismuthselinide{}}
