\documentclass[../mattg_ti-fi_lit-review.tex]{subfiles}
\textcolor{green}{For further information to include, read the following topical review article: } \cite{skomski_nanomagnetics_2003}

\subsection{Macroscopic magnetic classifications}
\subsubsection{Ferromagnetic}
Ferromagnetism is the strongest form of natural macroscopic magnetism, used in common everyday applications, such as refrigerator magnets or compass needles. There are commonly made up of compounds containing iron, cobalt, nickel or rare earth metals. There are two classes of ferromagnetic materials, ``soft'' and ``hard'', depending on their ability to have their microcrystalline structure polarised after manufacturing.

In 1948 Louis N\'eel showed that at a microscopic level, not only can ferromagnetism result from the alignment of magnetic moments but also can result from magnetisation where some moments point in the opposite direction (ferrimagnetism) \cite{neel_proprietes_1948}, however due to different contributions a net magnetisation remains.
\subsubsection{Anti-ferromagnetic}
As identified by N\'eel for ferrimagnetic materials, the magnetic ordering where neighbouring spins are misaligned can also cancel out any net magnetisation resulting in an anti-ferromagnetic material. However, above a particular temperature (dubbed the N\'eel temperature), the material undergoes a phase change and typically becomes paramagnetic due to the freedom for spins to be flipped.
There can be multiple ground states for spin systems, compared to the uniform case of a strict ferromagnet. This is referred to as a "geometric frustration", in that a system doesn't find a single ground state.
\subsubsection{Paramagnetic}
Paramagnetism is a material that induces a weak magnetic field in the direction of an applied field. It is due to the unpaired electrons in atoms with unfilled bands.
\subsubsection{Diamagnetic}
Diamagnetism on the other hand is the induction of an opposing magnetic field that repels the applied field. It is generally the result of static electrons that can screen changes in magnetic flux by acting like little current loops. It's effect is present in ferromagnetic and paramagnetic materials, but it's effect weak in comparison and so does not dominate the magnetisation dynamics.
Langevin diamagnetism and Landau diamagnetism govern the dynamics of localised and metallic electrons respectively.
\textcolor{red}{To fill out at a later date}

\subsection{Microscopic magnetic phenomena}
Magnetism has taken a while to be understood at its quantum, microscopic level, and still poses many questions today. Spin orbit coupling has already been discussed in Sec \ref{sec:SOC}, but here we will cover other phenomena that contribute to atomic scale magnetic phenomena.
\subsubsection{Quantum-mechanical exchange}
\textcolor{red}{To fill out at a later date}
\subsubsection{Crystal-field interaction}
\textcolor{red}{To fill out at a later date}

\subsection{Spintronics prior to topological insulators}
\textcolor{red}{To include at a later date, a history on spin effects in materials and the motivations towards spintronics pre-2000s, including some SOC further progress outside of TIs}. Checkout Rashba's review from 1965\cite{rashba_combined_1965}, and some more modern ones.

\subsection{Ferromagnetic Insualtors}
\textcolor{red}{NEED TO FILL OUT THIS SECTION ALOT} %TODO
Ferromagnetic insulators (FIs) are rare, because most magnetic materials that incorporate iron, cobalt or nickel are often conductive. That being said, some 3D FIs have been known for a while, and recently some new FIs that can be exfoliated down to thin flakes have been discovered.

\subsubsection{3D FIs}
\paragraph{BaFe$_{12}$O$_{19}$ (BFO)}  can be prepared by ALD, and can be grown to less than 5nm \cite{li_magnetization_2019}. 

\paragraph{Y$_3$Fe$_5$O$_{12}$ (YIG)}  is also a bulk film that can be grown thinly, for example down to 30nm in Wang \etal{} \cite{wang_surface-state-dominated_2016}.

\subsubsection{2D FIs}
\paragraph{Chromium triiodide (CrI$_3$)} was first reported on around 2014 \cite{mcguire_coupling_2015}. Recently reports show it's ferromagnetic properties down to few-layers \cite{huang_layer-dependent_2017, huang_electrical_2018-1}.
\begin{itemize}[noitemsep,topsep=0pt,parsep=0pt,partopsep=0pt]
	\item \url{https://pubs-acs-org.ezproxy.lib.monash.edu.au/doi/abs/10.1021/cm504242t?src=recsys }
	\item \url{https://www-nature-com.ezproxy.lib.monash.edu.au/articles/nature22391}
	\item \url{https://www-nature-com.ezproxy.lib.monash.edu.au/articles/s41565-018-0121-3 }
\end{itemize}

\paragraph{\cgt{} (CGT)} is also a new thin ferromagnetic insulator. It consists of quintuble players, similar to \bismuthtelluride{} or \bismuthselinide{}, and is also terminated with hexagonal Te planes. The discovery paper of 2017 \cite{gong_discovery_2017} shows that bilayers are stable for exfoliation. Whilst the bulk material has a Curie temperature (\textcolor{red}{todo - define the curie temperature}) of 61K, the 2D material is well defined below 40K. 

\paragraph{MgTiO$_3$ (MTO)}  is another recent addition to thin film ferromagnetic insulator films \cite{frantti_quest_2019}.

\textcolor{red}{Todo: discuss lattice matching in heterostructures page?} % TODO Lattice matching? 