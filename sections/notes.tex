\documentclass[../mattg_ti-fi_lit-review.tex]{subfiles}

\begin{document}
	\subsection{J. Moore - The Birth\cite{moore_birth_2010} [2010]}
	\subsubsection{Lessons from the past}
	\begin{itemize}
		\item Quantum Hall Effect can occur in 2D systems (condined) in the presence of a magentic field. It is a consequence of topological properties of the electronic wavefunctions. 
		\item Question - Can this effect arise without a large external magnetic field?
		\item In the 1980s, Predicted that forces from motion through crystal lattice could provide the same hall state \cite{haldane_model_1988}
		\item The mechanism which this recently has occured through is \textbf{spin-orbit coupling} or SOC. Spin and orbital angular momentum degrees of freedom are coupled. This coupling causes electrons to feel a spin dependent force.
		\item QSHE predicted in 2003. Unclear how to measure or if realistic or not. \cite{murakami_dissipationless_2003, murakami_spin-hall_2004}
		\item Kane \& Mele in 2005 produced a key theory advance, using realistic models. Showed that QSHE can survive by the use of invariants and could compute if the 2D material has an edge state or not. 2D Insulators that have 1D wires that conduct perfectly at low temps, similar to QHE. \cite{kane_z_2005}
		\item (Hg,Cd)Te quantum wells predicted to have quantized charge conductance  \cite{bernevig_quantum_2006}. These were then picked up in 2007 \cite{konig_quantum_2007}.
	\end{itemize}
	\subsubsection{Going 3D}
	\begin{outline}
		\1 2006 gave the revelation that while QHE doesn't go 2D to 3D, the TI does, subtly. \cite{fu_topological_2007, moore_topological_2007, roy_topological_2009}
		\1 "Weak 3D TI" given through stacking of multiple 2D TIs. Not stable to disorder. A dislocation will always conains a quantum wire at the edge.
		\1 "Strong 3D TI", connects ordinary insulators and topological insualtors by breaking time-reversal symmetry. \cite{moore_topological_2007} This is the focus result for experimental physics at the moment.
		\1 SOC is also required, but mixes all spins, unlike the 2D case. Spin direction dictates the momentum along the surface. Scattering also occurs, but metallic surface cannot vanish.
		\1 First 3D TI was Bi$_x$Sb$_{1-x}$. Surfaces mapped using ARPES experiment. The surface was complex, but launched a search for other TIs. \cite{hsieh_topological_2008,hsieh_observation_2009}.
		\1 Heavy metal, small bandgap semiconductors are meant to be the best candidates (this sounds like TMDs haha but not implied) for two reasons.
		\2 SOC is relativistic, and only strong for heavy elements.
		\2 Large bandgaps, relative to the scale of SOC, will not allow SOC to change the phase. This is because bands either need to invert or be able to cross or change properties by local deformation (I think).
		
		\1 The discovery of Bismuth Selenide Bi$_2$Se$_3$ and Bismuth Telluride Bi$_2$Te$_3$. They exhibit TI behaviour to higher temperatures, bulk bandgaps of $>$ 0.1 eV \cite{xia_observation_2009,zhang_topological_2009,chen_experimental_2009}.
		\2 The large bandgap is good for measurement in higher temperature
		\2 Simple surface states to investigate heterostructures etc. 
		\2 Complication - Distinction between surface and bulk states - residual conductivity resulting from impurities.
		
		\1 Graphene? \bismuthselinide{} looks like Graphene, due to it's Dirac nature. The difference for graphene is that a TI has only one Dirac point. Graphene has spin degeneracy with two Dirac points. The difference is major, including applications for quantum computing. This implies that the bandstructure \& reciporical lattice are really important for preserving coherent spin phenomena.
		
		\2 STMs found interference patterns near steps / defects, showing electrons aren't completely reflected - even when strong disorder occurs \cite{roushan_topological_2009,alpichshev_stm_2010, zhang_experimental_2009}. 
		\3 Generally, \textit{anderson localization}, the formation of insulating states, is meant to occur in materials with surface states (ie, Noble metals).
		\3 TI surface states are distinguished through this property! 
		\3 In some 2D models, it seems to form as a result of some disorder?
		\3 In graphene, apparently smooth potentials might do the same, but it's quite unlikely!
		\2 Femi level is located \textbf{at} the Dirac point for graphene! This is very adventageous, because of the tunability of the material. However, when it comes to other materials, it will not be as easy to tune the DOS around some point of interest like in graphene, without some form of doping. It was demonstrated recently in \bismuthselinide{} that you could tune to the Dirac point as well \cite{hsieh_tunable_2009}. The chemical potential is really important to be able to control for purposes like this. 	
		
	\end{outline}
	
	\subsubsection{Materials Challenges}
	\begin{outline}
		\1 Two important experiments:
		\2 Aharonov-Bohm Effect observed in nano-ribbons
		\2 MBE grow thin films of \bismuthselinide{} controlling thickness.
		\1 Items to characterize:
		\2 Transport \& optical properties on the metalic surface to measure the spin state and conductivity.
		\2 Improved materials with reduced bulk conductivity may be required!
	\end{outline}
	
	\subsubsection{A Nascent Field}
	\begin{outline}
		\1 Particle physics in the 1980s predicted a particle who coupled to ordinary EM fields.
		\1 Cond.Mat Physics seek phases of matter to respond to influence. 
		\2 Solid in response to sheer force
		\2 Superconductor in response to Meissner effect
		\2 T.I. in response to axion electrodynamics, ie in response to a scalar product $\textbf{E}.\textbf{B}$. (1980s)
		\3 Two coefficients correspond to OIs (ordinary) and TIs.
		\1 Just like insulators modify dielectric constant, coefficient of $\textbf{E}^2$ in Lagrangian, \\TI's generate non-zero coefficient of $\textbf{E.B}$
		\1 Applications are in electronics and magnetic memeory, not due to speed, but rather due to reproducability and no fatigue, as a result purely from the electrons.
		\1 Apparently, you can determine if a material is topological or not, purely from a polarizing magnetic field measurement.
		\1 The magnetoelectric effect can also create an electric field from a magnetic source, and vice versa. 
		
		\1 Some work also done to find similar phases, in materials with differing symmetries. Cooper pairs in superfluids ($^3$He) are closely connected to the quasiparticles/electrons in topological insulators. Direct experimental signatures would be awesome.
		\1 Moore goes on to talk about new phenomena, particularly at the interface of a superconductor and a topological insulator:
		
		\2 Creation of emergent particles
		\2 New particle statistics could be useful for 'topological' quantum computing which would protect from errors.
		\2 Fractional QHE also indicate new quasiparticle and statistics in materials.
		\2 TI surface becomes superconducting | Proximity effect
		\3 Vortex line from SC into the TI, a 0-energy Majorana fermion is trapped.
		\3 Majorana fermions have different quantum numbers to that of a regular electron.
		\3 Is roughly 'half' of an electron,  and it's own antiparticle.
		\2 Direct observation of Majorana fermions is a long sought goal
		\2 New particle statistics for other quasiparticles and systems. 
	\end{outline} 
	
	\subsection{Zhang - The Quantum Spin Hall Effect \& Topological Insulators \cite{qi_quantum_2010} - Physics Today (2010)}
	\subsubsection{Intro}
	\begin{outline}
		\1 Phases \& phenomena defined by symmetry	
		\2 Translation = Crystal Solids
		\2 Rotational = Magents
		\2 Gauge = Superconductors
		\1 QHE first example of non-spontaneously broken symmetry.
		\2 Topologically defined, not geometrically!
		\1 SQHE states are distinct again. 
		\2 Immune to impurities or geometric pertubations (momentum spin locking, scattering is supressed)
		\2 Maxwells equations altered by an additional 'topolgical' term, with remarkable effects. 
		\1 QHE differs to QSH
		\2 QHE External magnetic field required, TR symmetry broken
		\2 QSHE TR symmetric.
	\end{outline}
	\subsubsection{QH to QSH}
	\begin{outline}
		\1 Traffic control seperating out lanes of movement provides much less resistance. Same for QSH.
		\1 QHE Magnetic field pushes electrons to side lanes.
		\2 there are only two degrees of freedom in the system (forward above and back below)
		\2 There is no way for electrons to turn around
		\2 limited by external Magnetic Field requirement
		\1 In a real 1D system, you have four channels - spin up and down, moving forward and backward
		\2 These were predicted to be able to split into 2 separate chanels on two sides of the material. 
		\1 The backscattering supression is quantum mechanical.
		\2 Electrons in edge states can be thought to circle around the edge of the impurity and scatter backwards.
		\2 Depending on which way they rotate, they pick up a phase of $\pi$ or $-\pi$ (imagine a vector rotating around a circle). 
		\2 This induces a $2\pi$ phase difference between the two bodies. 
		\2 In QM, this corresponds to a negative sign. This means their amplitudes destructively interfere, leading to perfect transmission instead.
		\2 If the impurity is magnetic, TR symmetry is broken, and there is no longer destructive interference.
		\2 In an unseparated system (two movers back and foward) then spin doesn't need to change phase and regular scattering can take place.
		\2 The consequence then is, that there needs to be an "odd" number of forward and backward movers.
		\2 This is the heart of the $Z_2$ topological quantum number, and why a QSH insulator is also referred to as a topological insulator.
		\1 Requirements for 2D TIs
		\2 Heavy elements produce coupling of spin and orbital motion, relativistic.
		\1 Bernevig, Hughes and Zhang proposed a mechanism for finding TIs, nanoscopic layers sandwiched between other materials, become TIs after a critical thickness $d_c$ \cite{bernevig_quantum_2006}. This must be a pretty cool model, to achieve such a simple analytic result.
		\2 Mechanism is "band inversion", where general ordering of valence and coduction band is inverted by SOC.
		\2 Generally, \textit{s} orbitals contribute to conduction band, and valence is from the \textit{p} band.
		\2 In Hg and Te, SOC pushes S band above, and P band below.
		\2 Sandwiched between with Cadmium Telluride (similar lattice constant, but VERY different SOC).
		\2 Changing the thickness parameter changes the overall SOC of the quantum well. 
		
		\1 Discovered less than a year later \cite{konig_quantum_2007}. Clear difference from conduction quantum edge states at $2\frac{e^2}{h}$ to massive resistance (10M-Ohms) below that thickness.
	\end{outline}
	
	\subsubsection{2D to 3D}
	\begin{outline}	
		\1 2D insulator has a pair of 1D edge states crossing at momentum k=0. Linear dispersion here. Disperion in QFT from the Dirac Equation for a massless relativistic fermion in 1D, the equation is consequently used to describe the QSH state.
		\1 This can be generalised to a 3D system, with a 2D surface. Linear crossing becomes a Dirac cone. Crossing point is also at a TR-invariant point, such as at \textit{k}=0. Degeneracy is protected by TR symmetry, if you break TR then you break the degeneracy. 
		\1 Liang Fu and Kane predicted Bi$_{1-x}$Sb$_x$ to be a 3D TI \cite{fu_topological_2007}, observed by Hasan's group \cite{hsieh_topological_2008} using ARPES to measure the surface states.
		The authors of this review, Qi and Zhang \cite{qi_quantum_2010}, the predicted the states existent in \bismuthselinide{}, \bismuthtelluride{} and \antimonytelluride{} (note that \antimonyselenide{} is not) \cite{zhang_topological_2009}.
		\2 In \bismuthtelluride{} family, topology is again a result of band inversion between two orbitals, resulting from strong spin-orbit coupling of Bi and Te. By similarity, the family can also be described by a 3D version of the HgTe model which is awesome! 
		\2 First principle calculations show the Dirac cone.
		\2 Note that \bismuthselinide{} and \bismuthtelluride{} are both excellent thermoelectric materials as well.
		\2 The spin resolved measurements of Hasan's group with samples prepped by Robert Cava and co-workers managed to make this happen, showing spin lies in plane of surface, always perpendicular to the momentum 
		\2 Unfortunately the experiments also observed bulk carriers co-existing with surface states.
		\2 A pure TI phase without bulk carriers was observed in \bismuthtelluride{}, by Chen and Shen's group at Stanford \cite{chen_experimental_2009}, with materials prepped by Ian Fisher.
		
		\1 Models of TIs
		\2 Models are formed for QSHE by a Hamiltonian that is a Taylor expansion in the wavevector \textbf{k} of interactions between highest and lowest conduction bands.
		
		\3 HeTe (Mercury Telluride)
		\begin{equation}
		H(\textbf{k}) = \epsilon(k)\mathds{1} + \left(
		\begin{matrix}
		M(k) & A(k_x+ik_y) & 0 & 0\\
		A(k_x+ik_y) & -M(k) & 0 & 0\\
		0 & 0 & M(k) & -A(k_x+ik_y)\\
		0 & 0 & -A(k_x+ik_y) & -M(k)
		\end{matrix}
		\right)
		\end{equation}
		with \begin{equation}
		\epsilon(k)=C+Dk^2, M(k) = M - Bk^2
		\end{equation}
		
		\4 The upper 2x2 block describes the spin-up electrons in the s-like E1 conduction and p-like H1 valence bands.
		The lower 2x2 block describes the spin-down electrons in those same bands.
		\4 $\epsilon$ is the unimportant bending of all bands, where $\mathds{1}$ is the identity matrix.
		\4 Energy gap between the bands is 2M, and B describes the curvature of the bands. 
		\4 A incorporates the inter-band coupling at lowest order.
		\4 M/B < 0 has eigenstates of a trivial insulator.
		\4 M/B > 0, M becomes negative, and solution yields edge states of QSHE.
		\4 Can also practise doing this with a 2D TI honeycomb lattice to gain explicit understanding. \cite{kane_quantum_2005,kane_z_2005}
		
		\3 \bismuthtelluride{} (Bismuth Telluride) 
		\begin{equation}
		H(\textbf{k}) = \epsilon(k)\mathds{1} + \left(
		\begin{matrix}
		M(\textbf{k}) & A(k_x+ik_y) & 0 & A_1k_z\\
		A(k_x+ik_y) & -M(\textbf{k}) & A_1k_z & 0\\
		0 & A_1k_z & M(\textbf{k}) & -A(k_x+ik_y)\\
		A_1k_z & 0 & -A(k_x+ik_y) & -M(\textbf{k})
		\end{matrix}
		\right)
		\end{equation}
		with \begin{equation}
		\epsilon(k)=C+D_1k_z^2 + D_2k_\perp^2, M(k) = M - B_1k_z^2 - B_2k_\perp^2
		\end{equation}
		\4 This follows a similar model, in the context of bonding and anti-bonding $p_z$ orbitals with both spins. $B_1$ and $B_2$ have the same sign, and as before, depending on the sign of M, the bands undergo inversion.
		\2 It is essential to be in 3D to be able to construct a single Dirac cone for a 2D surface - the degeneracy of graphene with six Dirac cones makes it clear. The 2D HgTe quantum well at the crossover point $d=d_c$ also has two Dirac cones.  
	\end{outline}
	
	\subsubsection{TI Classification}
	\begin{outline}
		\1 Mathematicians group objects into broad classes. Topological invariants also created to classify.
		\1 For materials, Topological invariants can be found in topological field theories (TFT).
		\2 First classification: insulators that observe TR symmetry or not.
		\3 QH State breaks TR symmetry
		\3 QHS state does not break TR symmetry
		\2 Thouless et al. showed that physically measured QH conductance is given by an invariant, called \textbf{the first Chern number}. {See Physics TODAY- Aug2003, p38, Avron, Osadchy, Seiler}
		\2 Topological properties can be described by effective TFT, based on Chern-Simons theory.
		\1 Originally believes that 2D and TR breaking required for topological effects.
		\2 2001 first model of TR-invariant TI, in 4D.
		\2 Later reduced to 3D and 2D.
		\2 Theory developed for SHE and SOC identified as the missing ingredient for materials.
		\2 Theory extended to TR invariant TIs.
		\2 2D TR-Invariant QSH insulators shown to fall into two topological classes, through the $Z_2$ classification.
		\2 "Beautiful topological band theory" extended to 3D hahah
		\1 Now there are two precise definitions for TR-invariant TIs
		\2 Non-interacting topological band theory
		\3 Non interacting electrons filling a certain number of bands can calculate a binary value, $Z_2$.
		\2 Topological field theory
		\3 If with inversion symmetry, algorithm for electronic structure calculations can numerically evelaulte the topological band invariant.
		\1 It's important to have a definition for TIs that is valid for interacting systems (as insulators are interacting), whilst still being experimentally measurable. Turns out that topological field theory solves this.
		\1 This can be explained by elementary concepts in undergraduate electromagnetism:
		\2 Inside insulators, \textbf{E} and \textbf{B} are both well defined.
		\2 Can describe the electromagnetic response by an effective action.
		\2 This action is not topological, as it depends on a spatial geometry. This is seen in the fact that $\epsilon \textbf{E}^2-1/\mu\textbf{B}^2$ can be represented in the electromagnetic tensor form with indices. The summation over indices implies use of the metric tensor, which depends on the geometry (this leads to gravitational lensing of light?).
		\2 A different possible term is possible though, the spin-orbit coupling term $\textbf{E}.\textbf{B}$. It turns out that this term is independent of the metric and IS a topological term, because it only depends on the topology of the underlying space.
		\begin{align}
		S_\theta &= \frac{\theta\alpha}{4\pi^2}\int d^3xdt \textbf{E.B} \equiv \frac{\theta\alpha}{32\pi^2}\int d^3xdt \epsilon_{\mu\nu\rho\tau}F^{\mu\nu}F^{\rho\tau}\\
		&= \frac{\theta}{2\pi}\frac{\alpha}{4\pi}\int d^3xdt \partial^{\mu}\left(\epsilon_{\mu\nu\rho\tau}A^\nu\partial^\rho A^\tau\right)
		\end{align}
		
		\3 This value would naively break TR symmetry due to B, however in the periodic system, there are only two values of $\theta$, 0 or $\pi$. Phase doesn't change for these values, as if we were considering special values of flux, maintaining TR symmetry. 
		\3 Introducing a ferromagnetic layer on the boundary will also open up a bandgap under this theory, breaking TR on the surface but not in the bulk.
		\3 The last term in the equation above shows the bulk topological term is actually a total derivative expressed in a surface term in the parenthesis.
		\4 Specifies the hall conductance. $\theta=\pi$ translates to a half hall conductance $\frac{1}{2}e^2/h$. 
		\4 As additions can only change the quantum conductance by an interger multiple of $e^2/h$, the system is topologically robust, because this term can never be reduced to zero.		
	\end{outline}
	
	\subsubsection{Outlook}
	\begin{outline}
		\1 Interesting experiments:
		\2 Nanoribbons
		\2 Non-local measurements in HgTe devices to confirm QSH edge states.
		\1 Interesting theory:
		\2 Exotic excitations resulting from solving Maxwell's equations.
		\2 2D QSH insulator predicted to have fractional charge at edge, spin-charge separation in bulk.
		\2 A point charge above a material was expected to induce an image charge below. For a TI, this changes to also induce a magnetic monopole. This composite of electric and magnetic charges would be called a dyon, and would behave like an 'anyon', rather than obeying Bose or Fermi statistics.
		\2 A secondary system is expected to arise in the coupling of a SC system with a TI system, in the interface. The arrival of Majorana fermions.
		\2 It sounds like the magnetic monopoles are the methodology to be able to write magnetic memory, check out the citing article. (X.-L. Qi, Hughes, Zhang, Phys Rev. B, 195424, 2008, and ... Qi, science, 323, 11184 (2009))
		
	\end{outline}
	
\end{document}